\chapter{Conclusions and recommendations for future research}

\ifpdf
    \graphicspath{{Chapter7/figs/raster/}{Chapter7/figs/pdf/}{Chapter7/figs/}}
\else
    \graphicspath{{Chapter7/figs/vector/}{Chapter7/figs/}}
\fi
\section{Introduction}

This PhD has made advances in several aspects of multi-scale modelling of 
granular flows and understanding the complex rheology of dry and submerged 
granular flows. The significant contributions of this PhD are summarised in 
this chapter.

\subsection{Multi-scale modelling of dry granular flows}

A mult-scale approach is adopted to study the granular flow behaviour. The 
material point method, a 
continuum approach, is used to model the macro-scale response, while the 
grain-scale behaviour is captured using discrete element 
technique. In the 
present study, a two-dimensional DEM code is developed in C++ to study the 
micro-scale rheology of dry granular flows. A Verlet-list algorithm is 
implemented for neighbourhood detection to improve the computational 
efficiency. A linear-elastic model with a frictional contact behaviour is used 
to model dense rapid granular flows. A sweep-line Voronoi tesselation algorithm 
is implemented, in the present study, to extract continuum properties such as 
packing density from the local grain-scale simulations.

In order to capture the macro-scale response, a template-based 
three-dimensional C++ Material Point Method code, an Eulerian-Lagrangian 
approach, developed at the University of Cambridge is modified and extended to 
study granular flows as a continuum. In the present study, the Generalised 
Interpolation Material Point GIMP method is implemented to reduce the 
cell-crossing noise and oscillations observed during large-deformation 
problems, 
when using the standard MPM. The three-dimensional MPM code is parallelised to 
run on multi-core systems, thus improving the computational efficiency. The 
algorithm of the MPM code is improved to handle multi-body dynamics and 
interactions. Advanced constitutive models such as NorSand and 
modified Bingham fluid are also implemented. This dissertation includes 
only those results from two-dimensional plane-strain granular flow problems. 


\subsubsection*{Granular column collapse}

Multi-scale simulation of dry granular flows are performed to capture the 
local rheology, and to understand the capability and limitations of continuum 
models in realistic simulation of granular flow dynamics. For short columns, 
the run-out distance is found to be proportional to the granular mass 
destabilised above the failure surface. The spreading results from a 
Coulomb-like failure of the edges. The continuum approach, using a simple 
frictional dissipation model, is able to capture the flow dynamics of short 
columns. Unlike short columns, the collapse of tall columns is characterised by 
an initial collisional regime and a power-law dependence between the run-out 
and the initial aspect ratio of the granular column is observed. The energy 
evolution study reveals that the lack of collisional dissipation mechanism in 
the MPM simulations results in a substantially longer run-out distance for 
large aspect ratio columns. This 
shows that continuum approach using frictional laws are able to capture the 
flow kinematics at small aspect ratios, which is characterised by an inertial 
number \textit{I} less than 0.2 indicating a dense granular regime. However, 
a continuum approach like MPM is unable to precisely describe the flow 
dynamics of tall columns, which is characterised by an initial collisional 
regime (\textit{I} > 0.2). DEM studies on the role of initial material 
properties 
reveal that the initial packing fraction and the distribution of the kinetic 
energy in the system have a significant influence on the flow kinematics and 
the run-out behaviour. For the same material, a dense granular packing results 
in a longer run-out distance in comparison to the initially loose granular 
column. Hence, it is important to consider macroscopic parameters like packing 
fraction, which are due to meso-scale grain arrangements, when modelling the 
granular system as a continuum.

\subsubsection*{Granular slopes subjected to impact loading}

The ability of MPM in modelling transient flows that does not involve collision 
is further investigated. The distribution of kinetic energy in the granular 
mass is found to have a significant effect on the flow kinematics. In the 
present study, a multi-scale analyses of a granular slope subjected to 
impact velocities reveals a power-law dependence of the run-out distance and 
time as a function of the input energy with non-trivial exponents. The 
power-law behaviour is found to be a generic feature of granular dynamics. Two 
different regimes are observed depending on the input energy. The low energy 
regime reflects mainly the destabilisation of the pile, with a run-out time 
independent of the input energy. Whereas, the high energy regime involves 
spreading dynamics, which is characterised by a decay time that is defined as 
the time required for the input energy to  decline by a factor $1/2$. MPM is 
successfully able to simulate the transient evolution with a single 
input parameter, the macroscopic friction angle. This study exemplifies the 
suitability of MPM, as a continuum approach, in modelling large-deformation 
granular flow dynamics and opens the possibility of realistic simulations of 
geological-scale flows on complex topographies.


The distribution of the kinetic energy in the system is found to have a 
significant influence in the low energy regime, where a large 
fraction of the input energy is consumed in the destabilisation process. 
However at higher input energy, where most of the energy is dissipated during 
the spreading phase, the run-out distance has a weak dependence on 
the distribution of velocity in the granular mass. The material characteristics 
of the granular slope affect the constant of proportionality and not the 
exponent in the power-law relation between the run-out and the input energy. 

\subsection{Granular flows in fluid}

A two-dimensional coupled lattice Boltzmann - DEM technique is developed in C++ 
to understand the local rheology of granular flows in fluid. A multi-relaxation 
time LBM approach is implemented in the present study to ensure numerical 
stability. The coupled LBM--DEM technique offers the possibility to capture the 
intricate micro-scale effects such as the hydrodynamic instabilities. Coupled 
LBM-DEM involves modelling interactions of a few thousand soil grains with a 
few million fluid nodes. Hence, in the present study the LBM-DEM approach is 
implemented in the General Purpose Graphics Processing Units. The GPGPU 
implementation of the coupled LBM -- DEM technique offers the capability to 
model large scale fluid -- grain systems, which are otherwise impossible to 
simulate using conventional computational techniques. In the present study, 
simulations involving up to 5000 soil grains interacting with 9 million LBM 
fluid nodes are modelled. Efficient data transfer mechanisms that achieves 
coalesced global memory ensures that the GPGPU implementation scales linearly 
with the domain size. Granular flows in fluid involves soil grains interacting 
with fluid resulting in formation of turbulent vortices. In order to model the 
turbulent nature of granular flows, the LBM-MRT technique is coupled with the 
Smagorinsky turbulent model. The LBM-DEM code offers the possibility to 
simulate large-scale turbulent systems and probe micro-scale properties, which 
are otherwise impossible to capture in complex fluid - grain systems.

\subsubsection*{Granular collapse in fluid}

Two-dimensional LB-DEM simulations pose a problem of non-interconnected 
pore-space between the soil grains, which are in contact with each other. In 
the present study, a hydrodynamic radius, a reduction in the radius of the 
grains, is adopted during the LBM computation stage to ensure continuous 
pore-space for the fluid flow. A relation between the hydrodynamic radius and 
the permeability of the granular media is obtained. 

In order to understand the difference in the mechanism of granular flows in the 
dry and submarine conditions, LBM-DEM simulations of granular column collapse 
are performed and are compared with the dry case. Unlike the dry granular 
collapse, the run-out behaviour in fluid is found to be dictated by the initial 
volume fraction. For dense granular columns, the run-out distance in fluid is 
much shorter than its dry counterpart. Dense granular columns experience 
significantly high drag force and develop large negative pore-pressures during 
the initial stage of collapse resulting in a shorter run-out distance. On the 
contrary, granular columns with loose packing and low permeability tend to flow 
further in comparison to dry granular columns. This is due to entrainment of 
water at the flow front leading to hydroplaning. 

In both dense and loose initial packing conditions, the run-out distance is 
found to increase with decrease in the permeability. With decrease in 
permeability, the duration required for the flow to initiate takes longer due 
to the development of large negative pore-pressures. However, the low 
permeability of the granular mass results in entrainment of water at the flow 
front causing 
hydroplaning. For the same thickness and velocity of the flow, the potential of 
hydroplaning is influenced by the density of the flowing mass. Loose columns 
are more likely to hydroplane than the dense granular masses resulting in 
a longer run-out distance. This is in contrast to the behaviour observed in the 
dry collapse, where dense granular columns flow longer in comparison to loose 
columns.

Similar to the dry condition, a power-law relation is observed between the 
initial aspect ratio and the run-out distance in fluid. For a given 
aspect ratio and initial packing density, the run-out distance in the dry case 
is usually longer than the submerged condition. However, for the same kinetic 
energy, the run-out distance in fluid is found to be significantly higher than 
the dry conditions. The run-out distance in the granular collapse has a 
power-law relation with the peak kinetic energy. For the same peak kinetic 
energy, the run-out distance is found to increase with decrease in the 
permeability. The permeability, a material property, affects the constant of 
proportionality and not the exponent of the power-law relation between the 
run-out and the peak kinetic energy.


\subsubsection*{Granular collapse down inclined planes}

The influence of slope angle on the effect of permeability and the initial 
packing density on the run-out behaviour are studied. For increase in slope 
angle, the viscous drag on the dense column tends to predominate over the 
influence of hydroplaning on the run-out behaviour. The difference in the 
run-out between the dry and the submerged conditions, for a dense granular 
assembly, increases with increase in the slope angle above an inclination of 
5\si{\degree}. In contrast to the dense granular columns, the loose granular 
columns show a longer run-out distance in immersed conditions. The run-out 
distance increases with increase in the slope angle in comparison to the dry 
cases. The low permeable loose granular column retains the water entrained at 
the base of the flow front resulting in sustained lubrication effect. In 
contrast to the dry granular collapse, for all slope inclinations, the loose 
granular column in fluid flows further than the dense column. 

For granular collapse on inclined planes, the run-out distance is unaffected by 
the initial packing density at high permeability conditions. For collapse down 
inclined planes at high permeabilities, the viscous drag forces predominate 
resulting in almost the same run-out distance for both dense and loose initial 
conditions. However, at low permeability the entrainment of water at the flow 
front and the reduction in the effective stress of the flowing mass result in a 
longer run-out distance in the loose condition than the dense case with 
increase in the slope angle.

In tall columns, the run-out behaviour is found to be influenced by the 
formation of vortices during the collapse. The interaction of the surface 
grains with the surround fluid results in formation of vortices uniquely during 
the horizontal acceleration stage. The vortices result in redistribution of 
granular mass and thus affecting the run-out behaviour. This effect is 
predominant with increases in the slope angle. 

\section{Recommendations for future research}

Further research can be pursued along two directions: \textit{a}. improvement 
of the numerical tools and constitutive models to realistically simulate 
large-deformation problems and \textit{b}. investigation of the rheology of 
granular flows using experimental and numerical tools.

\subsection{Development of numerical tools}

\subsubsection*{Discrete element method}

The two-dimensional discrete element method, developed in the present study, 
can be extended to three-dimensions to model realistic soil flow 
problems. Although, linear-elastic contact model is found to be sufficient to 
describe rapid granular flows, further research using Hertz-Mindlin or other 
advanced contact model shall be performed. DEM is limited by the number of 
grains that can be simulated. Hence, it is important to be able to run DEM 
simulations on multi-core systems or on GPUs to model large-scale geometries. 
The initial gain properties are found to have a 
significant influence on the run-out behaviour, hence, it is vital to model 
grains of different shapes to understand their influence on the run-out 
distance. Agglomerates can also be used to study the effect of grain-crushing 
as the flow progresses down slope.

\subsubsection*{Material point method}

The present MPM code is capable of solving both 2D and 3D granular flow 
problems. Further research should focus on modelling three-dimensional granular 
flow problems and validate the suitability of MPM in modelling geological scale 
run-out behaviours. As the scale of the domain increases, the computational 
time increases especially when using GIMP method. To improve the computational 
efficiency, the material point method developed in the present study shall be 
modified to run on large clusters. The dynamic re-meshing 
technique~\citep{Shin2010a} shall be implemented to efficiently solve 
large deformation problems. The dynamic meshing approach is useful for problems 
involving motion of a finite size body in unbounded domains, in which the 
extent of material run-out and the deformation is unknown a \textit{priori}. 
The approach involves searching for cells that only contain material points, 
thereby avoiding unnecessary storage and computation. 

The current MPM code is capable of handling fluid-solid interactions in 
two-dimensions. Further research shall be pursued to implement fully coupled 3D 
material point method. The MPM code can also be extended to include the 
phase-transition behaviour in a continuum domain for partially fluidized 
granular flows~\citep{Aranson2002, Aranson2001, Volfson2003}. Fluid - solid 
interactions result in pressure oscillations. Further research is essential to 
explore advanced stabilisation methods that can be used to avoid the 
oscillations that occur due to incompressibility.

\subsubsection*{Lattice Boltzmann - DEM coupling}

The GPGPU parallelised 2D LBM-DEM coupled code, developed in the present study, 
shall be extended to 
three-dimensions. This would require a very high computational cost and hence 
it is important to parallelise the LBM-DEM code across multiple GPUs through a 
Message Passing Interface (MPI) similar to a large cluser parallelisation. A 
three-phase system of granular solids, water and air can be developed to 
realistically capture debris flow behaviour. The LB code can be extended to 
include a free surface, which can be used to investigate the influence of 
submarine mass movements on the free surface, such as tsunami generation.

\subsubsection*{Constitutive models}

DEM simulations of granular flow problems reveal that the initial material 
properties play a crucial role on the run-out evolution. The granular materials 
experience change in the packing fraction as the flow progresses. Hence, it is 
important to consider advanced models such as NorSand, a critical state based 
model, and $\mu(I)$  to model the dense granular flows. The behaviour of the 
soil under large deformations can be better expressed with a critical state 
model. Modified Nor-Sand constitutive model~\citep{Robert2010} implemented in 
the present study can be used in large-deformation flow problems. The $\mu(I)$ 
rheology, which is capable of capturing the complex rheology of dense granular 
flow, can be extended to include the effect of fluid 
viscosity~\citep{Pouliquen2005} to model granular flows in fluids. 


\subsection{Understanding the rheology of granular flows}

\subsubsection*{Granular column collapse}
Although, two-dimensional simulations provide a good understanding of the 
physics of granular flows, it is important to perform three-dimensional 
analysis to understand the realistic granular flow behaviour. Multi-scale 
simulations of three dimensional granular collapse experiments can be performed 
in dry and submerged conditions to understand the flow kinematics. Further 
research is essential to quantify the influence of initial packing density, 
shape and size of grains on the run-out behaviour for different initial aspect 
ratios. This would provide a basis for macro-scale parameters that are required 
to model the granular flow behaviour in a continuum scale.

\subsubsection*{Slopes subjected to impact}
This work may be pursued along two directions: \textit{a}. experimental 
realization of a similar setup with different modes of energy injection and 
\textit{b}. investigating the effect of various particle shapes or the presence 
of an ambient fluid. Although numerical simulations are generally reliable with 
realistic results found in the past studies of steady flows, the transient 
phases are more sensitive than steady flows and hence experimental 
investigation are necessary for validation. This configuration is also 
interesting for the investigation of the behaviour of a submerged slope 
subjected to an earthquake loading.

\subsubsection*{Granular flow down inclined planes}

Multi-scale analyses of large deformation flow problems such as the flow of dry 
granular materials down an inclined flume can be performed. This analysis will 
provide an insight on the limits of the continuum approach in modelling large 
deformation problems, which involve high shear-rates. The influence of 
parameters, such as particle size, density, packing and dilation, on the flow 
dynamics can be explored. These studies will be useful in describing the 
granular flow behaviour using the $\mu(I)$ rheology.

\subsubsection*{Granular flows in fluid}

Three dimensional LBM-DEM simulations of granular collapse in fluid can be 
carried out with varying shape, friction angle and size of particles to 
understand the influence of initial material properties on the run-out 
behaviour. Parametric analyses on the initial properties can be used to develop 
a non-dimensional number that is capable of delineating different flow regimes 
observed in granular flows in a fluid. Further research can be carried out on 
the collapse of tall columns and the influence of vortices on the run-out 
behaviour and re-distribution of the granular mass during the flow.



