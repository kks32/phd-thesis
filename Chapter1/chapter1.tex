\chapter{Introduction}

\ifpdf
    \graphicspath{{Chapter1/figs/raster/}{Chapter1/figs/pdf/}{Chapter1/figs/}}
\else
    \graphicspath{{Chapter1/figs/vector/}{Chapter1/figs/}}
\fi

\section{Introduction}
Avalanches, debris flows, and landslides are geophysical hazards which involve 
rapid mass movement of granular solids, water and air as a single phase system. 
The presence of water in a granular flow distinguishes `\textit{mud and debris 
flow}' from `\textit{granular avalanches}'. Debris flow is a rapid mass 
movement of liquefied, unconsolidated, saturated soil. The speed of the debris 
flow varies from 3 mph to 50 mph in extreme cases, transporting 100 to 100,000 
cubic meters of unconsolidated sediments down very steep slopes. 
Figure.~\ref{fig:debris} shows the debris flow that occurred during an 
earthquake-triggered landslide in Las Colinas,  El Salvador. Submarine 
landslides are marine landslides that transport sediments across continental 
shelves. Unlike a debris flow, the submarine landslides can occur on slopes as 
flat as $1^{o}$ and can reach speeds of 50 mph.~\cref{fig:Landslide} 
shows the Storegga Landslide, the largest recorded continental slope failure, 
which struck off the coast of central Norway, transporting materials over 300 
miles~\citep{Ward2002}. Granular avalanches, debris flow and submarine 
landslides cause significant damage to life and property. Globally, landslides 
cause billions of pounds in damage, and thousands of deaths and injuries each 
year. In July 2011, at least 32 people were killed in a debris flow induced by 
a rainfall-triggered landslide in South Korea (Source: BBC, 2011). 
Understanding the triggering mechanism and the granular flow process provides 
an insight into the force and velocity distribution in a granular flow, 
enabling us to design appropriate defensive measures. 

Granular flows have been extensively studied during the past two decades 
through experimental and numerical simulations~\citep{Jaeger1996}. In most 
cases, granular flows exhibit three distinct regimes: the slow quasi-static 
regime, a dilute collisional regime and an intermediate regime. Kinetic theory 
~\citep{Jenkins1983, Savage1981} which assumes binary collision between 
particles, captures the rapid-collisional regime, but is incapable of 
predicting the dense quasi-static behaviour. By drawing a simple analogy from 
fluid dynamics one can model granular flows as non-Newtonian fluids using a 
variant of the Navier-Stokes equation; one such approach is the depth-averaged 
shallow water equation~\citep{Savage1991}, which has been applied to solve 
granular flow dynamics with a reasonable amount of success. However, the basic 
assumption of neglecting the effect of vertical acceleration restricts the 
approach from describing the triggering mechanism, in which vertical 
acceleration plays a significant role. The dynamics of a homogeneous granular 
flow involve at least three distinct scales: the \textit{microscopic scale} 
which is characterized by contact between particles, the \textit{meso-scale} 
which represents micro-structural effects such as particle rearrangement, and 
the \textit{macroscopic scale}. A continuum description of granular flows is 
incapable of revealing inhomogeneities at particle level, such as force chains, 
which are purely due to micro-structural effects. Non-trivial relations have 
been found between the run-out distance and the initial aspect ratio of a 
granular column~\citep{Staron2007, Lajeunesse2005, Lube2005}. Simple 
mathematical models based on conservation of horizontal momentum capture the 
scaling laws of the final deposit, but fail to describe the initial transition 
regime. Rheology is concerned with describing the mechanical behaviour of those 
materials which cannot be described using classical theories, by establishing 
an empirical relation between deformations and stresses. 
Recently,~\citet{Midi2004} proposed a new rheology based on extensive 
experimental and numerical investigation on gravity-driven dry granular flows. 
The spreading dynamics are found to be similar for the continuum and particle 
approaches, however the rheology falls short in predicting the run-out distance 
for steeper slopes and in the transition regime where the shear-rate effect 
diminishes. The flow of granular materials in a fluid remains largely 
unexplored. 

\begin{figure}[htbp]
\centering
\includegraphics[width=0.95\textwidth]{Debrisflow}
\caption[Debris slide in Las Colinas, El Salvador, January 2001]{Initiation, 
channelling, spreading and deposition of debris slide in Las Colinas, El 
Salvador, January 2001. The debris flow buried as many as 500 homes. (Source: 
USGS report on `\textit{Landslides in Central America}', 2001)}
\label{fig:debris}
\end{figure}

Geophysical hazards usually involve multi-phase interactions. The momentum 
transfer between the discrete and continuous phases significantly affects the 
dynamics of the flow. Certain macroscopic models were unable to capture the 
complex physical mechanisms occurring at the particle scale, i.e. the 
hydrodynamic instabilities, formation of clusters, collapse, and transport 
phenomena. In order to describe the mechanism of multi-phase granular flows, it 
is important to consider both the dynamics of the solid phase and the role of 
the ambient fluid. Most models which simulate submarine landslides assume a 
single homogeneous grain-fluid mixture governed by a non-Newtonian fluid 
behaviour~\citep{Denlinger2001,Iverson2000}. Although successful in accounting 
for the general phenomenology on analytical grounds, such models fail to 
capture certain phenomena such as porosity gradient and fluid-induced 
size-segregation. Moreover, application of these models involves additional 
assumptions about the boundary/interface between the grains and the fluid, and 
the transition between high and low shear stresses. The $\mu(\textit{I})$ 
rheology is found to capture the dense submarine granular flows, if the 
inertial time scale in the rheology is replaced with a viscous time scale. 
However, the transition to the quasi-static regime where the shear rate 
vanishes is not captured by the simple model. The flow threshold or the 
hysteresis characterizing the flow or no-flow condition is also not correctly 
captured by the model. When the scale of the system is larger than the size of 
the structure, a simple rheology is expected to capture the overall flow 
behaviour, however the size of the correlated motion is of the same size as the 
system, causing difficulties in modelling the flow behaviour. Hence, it is 
essential to study the behaviour of granular flows at various scales, i.e. 
microscopic, meso-scale and continuum-scale levels, in order to develop a 
constitutive model that captures the entire flow process.

Typical continuum laws are only used when there is a strong separation of 
scales between the micro-scale and the macro-scale sizes of the flow geometry. 
Although granular materials are composed of discrete grains which interact only 
at contacts, the deformations of individual grains are negligible in comparison 
with the deformation of the granular assembly as a whole. Hence, the 
deformation is primarily due to the movements of grains as rigid bodies. 
Therefore, continuum models are still widely used to solve engineering problems 
associated with granular materials and flows. Conventional mesh-based 
approaches, such as Finite Element (FE) and Finite Difference (FD) methods, 
involve complex re-meshing and remapping of variables, which cause additional 
errors in simulating large deformation problems. Mesh-free methods, such as the 
Material Point Method (MPM) and Smooth Particle Hydrodynamics (SPH), are not 
constrained by the mesh size and its distortion, and hence are effective in 
simulating large deformation problems such as debris flow and submarine 
landslides. The analytical and finite element models which consider granular 
materials as a continuum cannot take into account the local geometrical 
processes that govern the mechanical behaviour of non-homogeneous soils, and 
pose subtle problems for statistical analysis. The particle level description 
of the granular material enriches the macro-scale variables that happen to 
poorly account for the local rheology of the materials. Numerical models based 
on the Discrete Element Method (DEM) allow us to evaluate quantities which are 
not accessible experimentally, thus providing useful insight into the flow 
dynamics. Particle-fluid interactions can now be simulated by interfacing 
discrete-element methods with a Lattice Boltzmann solver or a fluid-coupled 
continuum model, however both methods have their inherent limitations. Even 
though millions of particles can be simulated, the possible length of such a 
particle system is generally too small to regard it as `\textit{macroscopic}'. 
Therefore, methods to perform a micro-macro transition are important and these 
`\textit{microscopic}' simulations of a small sample, i.e. the 
`\textit{representative volume element}', can be used to derive a macroscopic 
constitutive relation which describes the material within the continuum 
framework. 

\begin{figure}[htpb]
\centering
\includegraphics[width=0.95\textwidth]{Storegga}
\caption[The extent of the Storegga landslide]{The extent of the Storegga 
landslide (Source: School of geoscience, University of Sydney)}
\label{fig:Landslide}
\end{figure}

\section{Objectives}
The aim of this research work is to understand and describe the mechanism of 
granular flows in dry and submerged conditions using a multi-scale approach. 
This research work involves identification of fundamental microscopic 
parameters that control the granular flow dynamics. Continuum and 
discrete-element modelling of granular flows will be performed to understand 
the limits of the continuum approach in modelling large deformation granular 
flow problems. This study will provide an insight into the mechanics of 
granular flows, and will enable us to understand geophysical hazards, such as 
avalanches, debris flows and sub-marine landslides. The study involves, but is 
not limited to, the following goals:

\tochide\subsubsection{Multi-scale simulation of dry granular flows}

Two-dimensional simulations of the collapse of a dry granular column and the 
flow of dry granular material on an inclined plane will be performed in 
particulate and continuum scales, using the Molecular Dynamics and Material 
Point Method, respectively. It involves understanding the parameters 
influencing the granular flow, i.e. the mechanism of flow initiation, flow 
propagation and the energy dissipation process. The study will highlight the 
limitations of the continuum and the microscopic approaches in capturing the 
mechanics behind the complex-flow problems.

\tochide\subsubsection{Constitutive modelling and fluid coupling in Material 
Point Method}

Advanced constitutive models, such as NorSand constitutive law and 
$\mu\left(\textit{I}\right)$ rheology models will be implemented in the 
Material Point Method. Large deformation plasticity theory and objective stress 
rates will implemented in the MPM code to describe the granular flow dynamics. 
The coupling of soil and fluid material points, which considers the dynamic 
fluid-solid interface will be developed. The effect of fluid pressure 
oscillations at the fluid-solid interface, will be studied. The fluid-coupled 
\textit{MPM} code will be implemented in a structured framework and robust 
visualization and parallelization algorithms will be implemented.

\tochide\subsubsection{Multi-scale simulation of saturated granular flows}

The Molecular Dynamics technique has been coupled with the Lattice Boltzmann 
method to simulate granular flows in a fluid domain. The validity of 
$\mu\left(\textit{I}\right)$ rheology in simulating granular flow in a fluid 
will be investigated. Continuum and particle level simulations of granular 
flows in fluids will be performed. The triggering mechanism, spreading dynamics 
and the energy dissipation mechanism between the dry and fluid saturated 
granular flows will be compared. The parameters influencing the granular 
collapse in fluids will also be investigated. 

\section{Thesis outline}
This thesis consists of seven chapters. Chapter one gives a brief introduction 
to geophysical hazards, such as submarine landslides, granular avalanches and 
debris flow. The behaviour of granular materials and the mechanism of granular 
flow is discussed in detail in chapter two, drawing attention to previous 
research on granular flow regimes. Experimental and numerical investigations on 
various granular flow problems and modelling techniques are also discussed in 
this chapter. By exploring different regimes in a granular flow and the 
limitations of various theoretical models, potential areas of improvement and 
future research in describing the mechanics of granular flow are identified. 
Continuum modelling of granular flow, the possibility of using a continuum 
model to describe the flow dynamics, and the limitations of a continuum 
approach are discussed in chapter three. Chapter three also elaborates on the 
Material Point Method, a Eulerian-Lagrangian approach, and its implementation 
in the present study to model granular flows. Chapter four explains the various 
discrete-element modelling techniques and explains the implementation of the 
Molecular Dynamics technique in modelling granular flow problems. The 
particle-assembling method adopted in the present study and the results of 
statistical analysis performed to verify the homogeneity of the prepared sample 
are also discussed in this chapter. Chapter five elaborates on the Lattice 
Boltzmann approach in modelling fluid flow and the coupling of the LBM approach 
with the Molecular Dynamics technique to simulate fluid-particle interactions. 
The validation of the Lattice Boltzmann technique in simulating fluid flow 
through a pipe with Computational Fluid Dynamics simulations is also discussed. 
The multi-scale simulations of granular column collapse using the Molecular 
Dynamics and the Material Point Method are discussed in chapter six. The final 
chapter summarizes the mechanics involved in a granular flow and outlines the 
goals and time-line of future research.