\chapter{Introduction}

\ifpdf
    \graphicspath{{Chapter1/figs/raster/}{Chapter1/figs/pdf/}{Chapter1/figs/}}
\else
    \graphicspath{{Chapter1/figs/vector/}{Chapter1/figs/}}
\fi

Avalanches, debris flows, and landslides are geophysical hazards which usually 
involve the rapid mass movement of granular solids, water and air as a 
multiphase system.
The presence of water in a granular flow distinguishes `\textit{mud and debris 
flow}' from `\textit{granular avalanches}'. Debris flow is a rapid mass 
movement of liquefied, unconsolidated, saturated soil. The speed of the debris 
flow varies from \SI{50}{\km\per\hour} to \SI{80}{\km\per\hour} in extreme 
cases, transporting 100 to 100,000 cubic meters of unconsolidated sediments 
down very 
steep slopes.~\Cref{fig:debris} shows the catastrophic effect of a debris flow 
that occurred during an earthquake-triggered landslide in Las Colinas,  El 
Salvador. On the other hand, submarine landslides transport sediments across 
continental shelves even on slopes as flat as \SI{1}{\degree} and can reach 
speeds of \SI{80}{\km\per\hour}.~\Cref{fig:Landslide} shows the Storegga 
Landslide, the largest recorded continental slope failure, which struck off the 
coast of central Norway, transporting materials over \SI{500}{\km} 
~\citep{Ward2002}. 

Granular avalanches, debris flow and submarine landslides cause significant 
damage to life and property. Globally, landslides cause billions of pounds in 
damage, and thousands of deaths and injuries each year. On 2 May 2014, a pair 
of mudslides killed at least 2000 people, burying 3000 houses and over 14,000 
people were affected after a landslide hit the north-east Afghan province of 
Badakhshan. Rescuers responding to the initial mudslide were struck by a second 
mudslide which trapped or killed a large proportion of potential rescuers 
(Source: BBC, 2014). The consecutive slides levelled the village, and left the 
area under 10 to 30 metres of mud. A week of torrential rain might be a 
plausible reason for the mud flows Understanding the triggering mechanism and 
the granular flow process provides an insight into the force and velocity 
distribution in a granular flow, enabling us to design appropriate defensive 
measures. 

\begin{figure}[tbhp]
\centering
\includegraphics[width=0.95\textwidth]{Debrisflow}
\caption[Debris slide in Las Colinas, El Salvador, January 2001]{Initiation, 
channelling, spreading and deposition of debris slide in Las Colinas, El 
Salvador, January 2001. The debris flow buried as many as 500 homes. (Source: 
USGS report on `\textit{Landslides in Central America}', 2001)}
\label{fig:debris}
\end{figure}

\begin{figure}[tbhp]
\centering
\includegraphics[width=0.95\textwidth]{Storegga}
\caption[The extent of the Storegga landslide]{The extent of the Storegga 
landslide (Source: School of geoscience, University of Sydney)}
\label{fig:Landslide}
\end{figure}

\section{Modelling granular flow}

The dynamics of a homogeneous granular flow involves at least three distinct 
scales: the \textit{microscopic scale} which is characterized by the contact 
between grains, the \textit{meso-scale} which represents micro-structural 
effects such as grain rearrangement, and the \textit{macroscopic scale}. The 
flow of submarine landslides, which can be as much as \SI{100,000}{\km\cubed} 
in volume, is 
influenced by the grain-grain interactions and the dynamics happening at the 
scale of a few micrometers to millimetres. This poses a question of how to 
effectively model the various scales of behaviour observed in a granular 
flow?

Typically, continuum laws are only used when there is a strong separation of 
scales between the micro-scale and the macro-scale sizes of the flow geometry. 
Although granular materials are composed of discrete grains which interact only 
at contacts, the deformations of individual grains are negligible in comparison 
with the deformation of the granular assembly as a whole. Hence, the 
deformation is primarily due to the movements of grains as rigid bodies. 
Therefore, continuum models are still widely used to solve engineering problems 
associated with granular materials and flows. 

Conventional mesh-based approaches, such as Finite Element (FE) and Finite 
Difference (FD) methods, involve complex re-meshing and remapping of variables, 
which cause additional errors in simulating large deformation problems. 
Mesh-free methods, such as the Material Point Method (MPM) and Smooth Particle 
Hydrodynamics (SPH), are not constrained by the mesh size and its distortion, 
and are effective in simulating large deformation problems such as debris 
flow and submarine landslides. The analytical and finite-element-like 
techniques which consider granular materials as a continuum cannot take into 
account the local geometrical processes that govern the mechanical behaviour of 
non-homogeneous soils, and pose subtle problems for statistical 
analysis~\citep{Mehta1994}. 

The grain level description of the granular material enriches the 
macro-scale variables that happen to poorly account for the local rheology of 
the materials. Numerical tools such as the Discrete Element Method (DEM) 
allow us to evaluate quantities which are not accessible experimentally, thus 
providing useful insight into the flow dynamics. Grain-fluid interactions can 
be simulated by interfacing discrete-element methods with a lattice 
Boltzmann solver or a computational fluid dynamics solver, however these 
methods have their inherent limitations. Even though millions of grains can be 
simulated, the possible size of such a grain system is generally too small 
to regard it as `\textit{macroscopic}'. Therefore, methods to perform a 
micro-macro transition are important and these `\textit{microscopic}' 
simulations of a small sample, i.e. the `\textit{representative volume 
element}', can be used to derive a macroscopic theories which describes the 
material within the continuum framework. 

Granular flows have been extensively studied during the past two decades 
through experimental and numerical 
simulations~\citep{Jaeger1996,Iverson1997a,Denlinger2001,Tang2013,Andersen2010}.
In most cases, granular flows exhibit three distinct regimes: the slow 
quasi-static regime, a dilute collisional regime and an intermediate regime. 
Many theories and phenomenological models have been developed to model the 
behaviour of different flow regimes. One approach is to use Kinetic 
theory~\citep{Jenkins1983, Savage1981}, which assumes binary collision between 
particles. Kinetic theory is able to capture the rapid-collisional regime, 
however is incapable of predicting the dense quasi-static behaviour. Under 
certain conditions, granular 
flows exhibit some fluid-like behaviour, using a simple analogy from fluid 
dynamics one can model granular flows as non-Newtonian fluids using a variant 
of the Navier-Stokes equation~\citep{Savage1991}. The depth-averaged shallow 
water equation has been applied to solve 
granular flow dynamics with a reasonable amount of success. However, the basic 
assumption of neglecting the effect of vertical acceleration restricts the 
approach from describing the triggering mechanism, such as the collapse of a 
vertical cliff.

In certain cases, classical theories are incapable of describing the flow 
kinematics. Hence, rheologies have been used to describe the mechanical 
behaviour of granular flows through an empirical relation between deformations 
and stresses.~\citet{Midi2004} proposed a new rheology for granular flows based 
on extensive experimental and numerical investigation on gravity-driven flows. 
The $\mu(I)$ rheology describes the granular behaviour using a dimensionless 
number, called the \textit{inertial number I}, which is the ratio of inertia to 
the pressure forces. Small values of \textit{I} correspond to the quasi-static 
regime, and large values of \textit{I} corresponds to the fully 
collisional regime of the kinetic theory. The spreading dynamics are found to 
be similar for the continuum and grain-scale approaches, however the rheology 
falls short in predicting the run-out distance for steeper slopes and in the 
transition regime where the shear-rate effect diminishes. The use of rheologies 
to describe the flow of granular materials in fluids remains largely 
unexplored.


In addition to the scale-effects, most geophysical hazards usually involve 
multi-phase interactions. The momentum transfer between the discrete and 
continuous phases significantly affects the dynamics of the flow. In order to 
describe the mechanism of multi-phase granular flows, it is important to 
consider both the dynamics of the solid phase and the role of the ambient 
fluid. Most models which simulate submarine landslides assume a single 
homogeneous grain-fluid mixture governed by a non-Newtonian fluid 
behaviour~\citep{Denlinger2001,Iverson2000}. Although successful in accounting 
for the general phenomenology on analytical grounds, such models fail to 
capture certain phenomena such as porosity gradient and fluid-induced 
size-segregation. Moreover, application of these models involves additional 
assumptions about the boundary/interface between the grains and the fluid, and 
the transition between high and low shear stresses. 

The simple $\mu(\textit{I})$ rheology is found to capture the dense submarine 
granular flows, if the inertial time scale in the rheology is replaced with a 
viscous time scale~\citep{Pouliquen2005}. However, the transition from a rapid 
granular flow down a slope to the quasi-static regime when the granular mass 
ceases to flow, where the shear rate decreases rapidly, is not captured by the 
simple model. The flow threshold or the hysteresis characterizing the flow or 
no-flow condition is also not correctly captured by this model. When the scale 
of the system is larger than the size of the structure, a simple rheology is 
expected to capture the overall flow behaviour, however in granular flows, the 
size of the correlated motion has the same size as the system, causing 
difficulties in modelling the flow behaviour. Hence, it is essential to study 
the behaviour of granular flows at various scales, i.e. microscopic, meso-scale 
and continuum-scale levels, in order to describe the entire granular flow 
process.


\section{Objectives}

This study is motivated by a simple question: If a granular column collapses 
and flows in a dry condition and within a fluid, in which case would the 
run-out be the farthest? The collapse in a fluid experiences drag forces which 
tend to retard the flow, this might result in a longer run-out distance in the 
case of dry condition. However, the collapse in fluid might experience 
lubrication effects due to hydroplaning and this reduces the effective 
frictional resistance, which might result in a longer run-out distance in the 
case of fluid than the dry condition. Would the effect of lubrication overcome 
the drag forces and result in the farthest run-out in fluid? Or would the drag 
forces predominate resulting in dry case having the farthest run-out 
distance? Although, a simple problem by description, the influence of various 
properties such as slope angle, initial packing density, and permeability on 
the run-out behaviour makes it a complex phenomenon.

The other important question is how to model the granular flow behaviour, which 
exhibits complex fluid-like and/or solid-like behaviour depending on the 
initial state and the ambient conditions. This research aims to 
provide an insight into the mechanics of granular flows in dry and 
submerged conditions, using a multi-scale approach, so as to describe the 
behaviour of geophysical hazards such as avalanches, debris flows and 
sub-marine landslides.

\section{Overview of this work}

This PhD makes advances in the field of numerical modelling of granular flows. 
Granular materials exhibit complex flow behaviour that often involves 
multiphase interactions. Development of sophisticated numerical tools that are 
capable of modelling the different scales of description in a granular flow and 
the multiphase interaction between the soil grains and the fluid is thereby the 
focus of this PhD. This study addresses the following 
areas of granular flow modelling and behaviour:
%
\begin{description}
\item[Development of numerical tools for modelling dry granular flows (Chapter 
3)]{The Material Point Method, a continuum based Eulerian - Lagrangian 
approach, is developed to describe the dynamics of the various granular flow 
behaviours (described in Chapter 2). The grain-scale response of the granular 
flow kinematics is captured using Discrete Element Method (DEM). The 
implementation of DEM and development of tessellations tools for extracting 
macro-scale properties are also discussed.}

\item[Dry granular flows (Chapter 4)] {Multi-scale analyses using MPM and DEM 
are performed to understand the suitability of MPM as a continuum approach in 
modelling collapse of a granular column. Energy evolution studies are carried 
out to study the difference in the dissipation mechanism between both 
approaches. The influence of various initial material properties on the run-out 
behaviour is investigated. The run-out behaviour of a granular slope subjected 
to a horizontal excitation is analysed using MPM and DEM. The effect of 
distribution of kinetic energy on the run-out behaviour of granular slopes is 
also discussed.}

\item[Numerical modelling of fluid - grain coupling (Chapter 5)]{A coupled 
lattice Boltzmann - DEM approach is implemented in this work to understand the 
micro-scale interactions between the soil grains and the ambient fluid. Due to 
the computational demand of this approach, the LBM - DEM technique is 
implemented in GPUs. A turbulent model is incorporated to capture the
interaction of vortices with the granular surface. The validation of 
the developed LBM-DEM model under different flow conditions is also discussed.}

\item[Granular flows in fluid (Chapter 6)]{The fundamental question of the 
difference in the mechanism of collapse and flow between the dry and the 
submerged granular columns is addressed. The role of initial granular 
properties such as packing density, permeability and slope angle on the run-out 
behaviour of the dry and the submerged granular flows is investigated. This 
chapter provides an insight into the flow dynamics of submerged granular flows 
under different initial conditions.
}
\end{description}

The videos of simulations performed in this study can be viewed 
at~\url{http://vimeo.com/kks32/videos}. 

\nomenclature[z-DEM]{DEM}{Discrete Element Method}
\nomenclature[z-FEM]{FEM}{Finite Element Method}
\nomenclature[z-FVM]{FVM}{Finite Volume Method}
\nomenclature[z-BEM]{BEM}{Boundary Element Method}
\nomenclature[z-MPM]{MPM}{Material Point Method}
\nomenclature[z-LBM]{LBM}{Lattice Boltzmann Method}
\nomenclature[z-MRT]{MRT}{Multi-Relaxation 
Time}
\nomenclature[z-RVE]{RVE}{Representative Elemental Volume}
\nomenclature[z-GPU]{GPU}{Graphics Processing Unit}
\nomenclature[z-SH]{SH}{Savage Hutter}
\nomenclature[z-CFD]{CFD}{Computational Fluid Dynamics}
\nomenclature[z-LES]{LES}{Large Eddy Simulation}
\nomenclature[z-FLOP]{FLOP}{Floating Point Operations}
\nomenclature[z-ALU]{ALU}{Arithmetic Logic Unit}
\nomenclature[z-FPU]{FPU}{Floating Point Unit}
\nomenclature[z-SM]{SM}{Streaming Multiprocessors}
\nomenclature[z-PCI]{PCI}{Peripheral Component Interconnect}
\nomenclature[z-CK]{CK}{Carman - Kozeny}
