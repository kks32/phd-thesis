\chapter{Introduction}

\ifpdf
    \graphicspath{{Chapter1/figs/raster/}{Chapter1/figs/pdf/}{Chapter1/figs/}}
\else
    \graphicspath{{Chapter1/figs/vector/}{Chapter1/figs/}}
\fi

Avalanches, debris flows, and landslides are geophysical hazards which involve 
rapid mass movement of granular solids, water and air as a single phase system. 
The presence of water in a granular flow distinguishes `\textit{mud and debris 
flow}' from `\textit{granular avalanches}'. Debris flow is a rapid mass 
movement of liquefied, unconsolidated, saturated soil. The speed of the debris 
flow varies from 50~\si{\km\per\hour} to 80~\si{\km\per\hour} in extreme cases, 
transporting 100 to 100,000 cubic meters of unconsolidated sediments down very 
steep slopes.~\Cref{fig:debris} shows the catastrophic effect of a debris flow 
that occurred during an earthquake-triggered landslide in Las Colinas,  El 
Salvador. On the other hand, Submarine landslides transport sediments across 
continental shelves even on slopes as flat as 1\si{\degree} and can reach 
speeds of 80~\si{\km\per\hour}.~\Cref{fig:Landslide} shows the Storegga 
Landslide, the largest recorded continental slope failure, which struck off the 
coast of central Norway, transporting materials over 500~\si{\km} 
~\citep{Ward2002}. 

Granular avalanches, debris flow and submarine landslides cause significant 
damage to life and property. Globally, landslides cause billions of pounds in 
damage, and thousands of deaths and injuries each year. On 2 May 2014, a pair 
of mudslides killed at least 2000 people, 3000 houses were buried and over 
14,000 affected after a landslide hit the north-east Afghan province of 
Badakhshan. Rescuers responding to the initial mudslide were struck by a second 
mudslide which trapped or killed a large proportion of potential rescuers, 
(Source: BBC, 2014). The consecutive slides levelled the village, and left the 
area under 10 to 30 metres of mud. A week of torrential rain might be a 
plausible reason for the mudflows. Understanding the triggering mechanism and 
the granular flow process provides an insight into the force and velocity 
distribution in a granular flow, enabling us to design appropriate defensive 
measures. 

\begin{figure}[tbhp]
\centering
\includegraphics[width=0.95\textwidth]{Debrisflow}
\caption[Debris slide in Las Colinas, El Salvador, January 2001]{Initiation, 
channelling, spreading and deposition of debris slide in Las Colinas, El 
Salvador, January 2001. The debris flow buried as many as 500 homes. (Source: 
USGS report on `\textit{Landslides in Central America}', 2001)}
\label{fig:debris}
\end{figure}

\begin{figure}[tbhp]
\centering
\includegraphics[width=0.95\textwidth]{Storegga}
\caption[The extent of the Storegga landslide]{The extent of the Storegga 
landslide (Source: School of geoscience, University of Sydney)}
\label{fig:Landslide}
\end{figure}

\section{Studies on granular flows}

The dynamics of a homogeneous granular flow involve at least three distinct 
scales: the \textit{microscopic scale} which is characterized by the contact 
between grains, the \textit{meso-scale} which represents micro-structural 
effects such as grain rearrangement, and the \textit{macroscopic scale}. In a 
submarine landslides, which is of 100,000 \si{\km\cubed} in volume is 
influenced by the grain-scale dynamics happening at the scale of 
a few $\mu$ meters to millimetres. This poses a question of how to 
appropriately model the various scales of behaviour observed in a granular 
flow. 

Typically, continuum laws are only used when there is a strong separation of 
scales between the micro-scale and the macro-scale sizes of the flow geometry. 
Although granular materials are composed of discrete grains which interact only 
at contacts, the deformations of individual grains are negligible in comparison 
with the deformation of the granular assembly as a whole. Hence, the 
deformation is primarily due to the movements of grains as rigid bodies. 
Therefore, continuum models are still widely used to solve engineering problems 
associated with granular materials and flows. 

Conventional mesh-based approaches, such as Finite Element (FE) and Finite 
Difference (FD) methods, involve complex re-meshing and remapping of variables, 
which cause additional errors in simulating large deformation problems. 
Mesh-free methods, such as the Material Point Method (MPM) and Smooth Particle 
Hydrodynamics (SPH), are not constrained by the mesh size and its distortion, 
and hence are effective in simulating large deformation problems such as debris 
flow and submarine landslides. The analytical and finite-element-like 
techniques which consider granular materials as a continuum cannot take into 
account the local geometrical processes that govern the mechanical behaviour of 
non-homogeneous soils, and pose subtle problems for statistical analysis. 

The grain level description of the granular material enriches the 
macro-scale variables that happen to poorly account for the local rheology of 
the materials. Numerical models based on the Discrete Element Method (DEM) 
allow us to evaluate quantities which are not accessible experimentally, thus 
providing useful insight into the flow dynamics. Grain-fluid interactions can 
be simulated by interfacing discrete-element methods with a Lattice 
Boltzmann solver or a Computational Fluid Dynamics solver, however these 
methods have their inherent limitations. Even though millions of grains can be 
simulated, the possible length of such a grain system is generally too small to 
regard it as `\textit{macroscopic}'. Therefore, methods to perform a 
micro-macro transition are important and these `\textit{microscopic}' 
simulations of a small sample, i.e. the `\textit{representative volume 
element}', can be used to derive a macroscopic theories which describes the 
material within the continuum framework. 

Granular flows have been extensively studied during the past two decades 
through experimental and numerical 
simulations~\citep{Jaeger1996,Iverson1997a,Denlinger2001,Tang2013,Andersen2010}.
In most cases, granular flows exhibit three distinct regimes: the slow 
quasi-static regime, a dilute collisional regime and an intermediate regime. 
Many theories and phenomenological models have been developed to model the 
behaviour the different flow regimes. One approach is to use the Kinetic 
theory,~\citep{Jenkins1983, Savage1981} which assumes binary collision between 
particles. Kinetic theory is able to capture the rapid-collisional regime, 
however is incapable of predicting the dense quasi-static behaviour. Granular 
flows exhibit fluid like behaviour, using a simple analogy from fluid 
dynamics one can model granular flows as non-Newtonian fluids using a variant 
of the Navier-Stokes equation; one such approach is the depth-averaged shallow 
water equation~\citep{Savage1991}. This approach has been applied to solve 
granular flow dynamics with a reasonable amount of success. However, the basic 
assumption of neglecting the effect of vertical acceleration restricts the 
approach from describing the triggering mechanism, in which the vertical 
acceleration plays a significant role such as collapse of a vertical cliff.

In certain cases, classical theories are incapable of describing the flow 
kinematics. Hence, rheologies have been used to describe the mechanical 
behaviour of granular flows through an empirical relation between deformations 
and stresses.~\citet{Midi2004} proposed a new rheology for granular flows based 
on extensive experimental and numerical investigation on gravity-driven flows. 
The $\mu(I)$ rheology describes the granular behaviour using a dimensionless 
number, called the \textit{inertial number I}, which is the ratio of inertia to 
the pressure forces. Small values of \textit{I} correspond to the critical 
state in soil mechanics and large values of \textit{I} corresponds to the fully 
collisional regime of kinetic theory. The spreading dynamics are found to be 
similar for the continuum and grain-scale approaches, however the rheology 
falls short in predicting the run-out distance for steeper slopes and in the 
transition regime where the shear-rate effect diminishes. The flow of granular 
materials in a fluid remains largely unexplored. 


In addition to the scale-effects, most geophysical hazards usually involve 
multi-phase interactions. The momentum transfer between the discrete and 
continuous phases significantly affects the dynamics of the flow. In order to 
describe the mechanism of multi-phase granular flows, it is important to 
consider both the dynamics of the solid phase and the role of the ambient 
fluid. Most models which simulate submarine landslides assume a single 
homogeneous grain-fluid mixture governed by a non-Newtonian fluid 
behaviour~\citep{Denlinger2001,Iverson2000}. Although successful in accounting 
for the general phenomenology on analytical grounds, such models fail to 
capture certain phenomena such as porosity gradient and fluid-induced 
size-segregation. Moreover, application of these models involves additional 
assumptions about the boundary/interface between the grains and the fluid, and 
the transition between high and low shear stresses. 

The simple $\mu(\textit{I})$ rheology is found to capture the dense submarine 
granular flows, if the inertial time scale in the rheology is replaced with a 
viscous time scale~\citep{Pouliquen2005}. However, the transition from a rapid 
granular flow down a slope to the quasi-static regime when the granular mass 
ceases to flow, where the shear rate vanishes, is not captured by the simple 
model. The flow threshold or the hysteresis characterizing the flow or no-flow 
condition is also not correctly captured by this model. When the scale of the 
system is larger than the size of the structure, a simple rheology is expected 
to capture the overall flow behaviour, however in granular flows, the size of 
the correlated motion is of the same size as the system, causing difficulties 
in modelling the flow behaviour. Hence, it is essential to study the behaviour 
of granular flows at various scales, i.e. microscopic, meso-scale and 
continuum-scale levels, in order to describe the entire granular flow process.


\section{Objectives}

This study is motivated by a simple question: If a granular column collapses 
and flows in a dry condition and within a fluid, in which case would the 
run-out be the farthest? The collapse in a fluid experiences drag forces which 
tend to retard the flow, this might result in a longer run-out distance in the 
case of dry condition. However, the collapse in fluid might experience 
lubrication effects due to hydroplaning and this reduces the effective 
frictional resistance, which might result in a longer run-out distance in the 
case of fluid than the dry condition. Would the effect of lubrication overcome 
the drag forces and result in the farthest run-out in fluid? or would the drag 
forces predominate resulting in dry condition having the farthest run-out 
distance. Although, a simple problem by description, the influence of various 
properties such as slope angle, initial packing density, permeability on the 
run-out behaviour makes it a complex phenomenon.

The other important question is how to model the granular flow behaviour, which 
exhibits complex fluid-like and/or solid-like behaviour depending on the 
initial state and the ambient conditions. Despite advances in numerical tools, 

The aim of this research work is to 
understand and describe the mechanism of granular flows in dry and submerged 
conditions using a multi-scale approach. This research work involves 
identification of fundamental microscopic parameters that control the granular 
flow dynamics. Continuum and discrete-element modelling of granular flows are 
performed to understand the limits of the continuum approach in modelling 
large deformation granular flow problems. This study provides an insight 
into the mechanics of granular flows, and enables us to understand the 
mechanisms involved in geophysical hazards, such as avalanches, debris flows 
and sub-marine landslides.

