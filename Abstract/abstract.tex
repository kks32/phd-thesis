% ************************** Thesis Abstract *****************************
% Use `abstract' as an option in the document class to print only the titlepage and the abstract.
\begin{abstract}


Geophysical hazards usually involve multiphase flow of dense granular solids 
and water. Understanding the mechanics of granular flow is of 
particular importance in predicting the run-out behaviour of debris flows. The 
dynamics of a homogeneous granular flow involve three distinct scales: 
the microscopic scale, the meso-scale, and the macroscopic scale. 
%
%the \textit{microscopic scale}, which is characterised by the contact between 
%grains, the \textit{meso-scale} that represents micro-structural effects such 
%as grain rearrangement, and the \textit{macroscopic scale}, where geometric
%correlations can be observed. 
%
Conventionally, granular flows are modelled as a continuum because 
they exhibit many collective phenomena. 
%However, on a grain scale, the granular 
%materials exhibit complex solid-like and/or fluid-like behaviour.
Recent studies, however, suggest that a continuum law may be unable to capture 
the effect of inhomogeneities at the grain scale level, such as orientation of 
force chains, which are micro-structural effects. Discrete element methods 
(DEM) are capable of simulating these micro-structural effects, however they 
are computationally expensive. In the present study, a multi-scale approach is 
adopted, using both DEM and continuum techniques, to better understand the 
rheology of granular flows and the limitations of continuum models.

The collapse of a granular column on a horizontal surface is a simple case of 
granular flow; however, a proper model that describes the flow dynamics is still
lacking. 
%Granular flow is modelled as a frictional dissipation process in 
%continuum mechanics but studies showing the lack of influence of 
%inter-particle 
%friction on the energy dissipation and spreading dynamics is surprising. 
In the present study, the generalised interpolation material point method 
(GIMPM), a hybrid Eulerian -- Lagrangian approach, is implemented with the
Mohr-Coloumb failure criterion to describe the continuum behaviour of granular 
flows. The granular column collapse is also simulated using DEM to understand 
the micro-mechanics of the flow.
%
%The run-out distance of a granular column collapse exhibits a power law 
%dependency with the aspect ratio of the column.
The limitations of MPM in modelling the flow dynamics are 
studied by inspecting the energy dissipation mechanisms. The lack 
of collisional dissipation in the Mohr-Coloumb model results in longer run-out 
distances for granular flows in dilute regimes (where the mean pressure is 
low). However, the model is able to capture the rheology of dense granular 
flows, such as the run-out evolution of slopes subjected to lateral excitation, 
where the inertial number \textit{I} < 0.1. 

The initiation and propagation of submarine flows 
depend mainly on the 
slope, density, and quantity of the material destabilised. Certain macroscopic 
models are able to capture simple mechanical behaviours, however the complex 
physical mechanisms that occur at the grain scale, such as hydrodynamic 
instabilities and formation of clusters, have largely been ignored. In order to 
describe the mechanism of submarine granular flows, it is 
important to consider both the dynamics of the solid phase and the role of the 
ambient fluid. In the present study, a two-dimensional coupled Lattice 
Boltzmann LBM -- DEM technique is developed to understand the micro-scale 
rheology of granular flows in fluid. Parametric analyses are performed to 
assess the influence of initial configuration, permeability, and slope 
of the inclined plane on the flow. The effect of hydrodynamic forces on the 
run-out 
evolution is analysed by comparing the energy dissipation and flow 
evolution between dry and immersed conditions.
\end{abstract}
