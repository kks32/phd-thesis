% ************************** Thesis Abstract *****************************
% Use `abstract' as an option in the document class to print only the titlepage and the abstract.
\begin{abstract}


Geophysical hazards usually involve flow of dense granular solids and water as 
a single-phase system. The dynamics of a homogeneous granular flow involve 
three distinct scales: the \textit{microscopic scale}, the \textit{meso-scale}, 
and the \textit{macroscopic scale}. Although soil flows are conventionally 
modelled as a continuum, recent studies have shown the limitations of continuum 
models to capture the in-homogeneities at the grain-scale level. On a grain 
scale, the granular materials exhibit complex solid-like and/or fluid-like 
behaviour depending on how the grains interact with each other. In the present 
study, a multi-scale approach is adopted to understand the rheology of granular 
flows and the limitation of continuum models.

The Generalised Interpolation Material Point Method (GI-MPM), a  hybrid 
Eulerian -- Lagrangian approach is implemented in the present study to describe 
the continuum behaviour of granular flows. The Discrete Element Method (DEM) is 
used to model the micro-mechanics of granular flows. The two-dimensional 
plane-strain collapse of a granular column on a horizontal surface is analysed 
both at the continuum scale and at the grain scale.

The run-out distance of a granular column collapse exhibits a power law 
dependency with the aspect ratio of the column. For columns with small aspect 
ratios (`h/l' $\le$ 2), both MPM and DEM approaches predict similar run-out 
behaviour. However, MPM predicts longer run-out distances for columns with 
larger aspect ratios (`h/l' > 2). Analysis the energy evolution during the 
collapse reveals higher collisional dissipation in the initial free-fall regime 
for tall columns. The lack of collisional energy dissipation mechanism in MPM 
simulations results in longer run-out distances. The dimensionless inertial 
number \textit{I} is used to identify different flow regimes during a column 
collapse. The classical Mohr-Coloumb model has the ability to capture the 
rheology of granular flows  in dense-granular and critical state flow regimes 
(for example, run-out evolution of slopes subjected to impact loading, where 
the Inertial number \textit{I} < 0.1). Voronoi tesselation is used to study the 
meso-scale behaviour such as the evolution of local packing density for 
different initial volume fraction.  

Certain macroscopic models are able to capture simple mechanical behaviours, 
however the complex physical mechanisms that occur at the grain scale, such as 
hydrodynamic instabilities, the formation of clusters, collapse, and transport, 
have largely been ignored. In order to describe the mechanism of saturated 
and/or immersed granular flows, it is important to consider both the dynamics 
of the solid phase and the role of the ambient fluid. In particular, when the 
solid phase reaches a high volume fraction, it is important to consider the 
strong heterogeneity arising from the contact forces between the grains, the 
drag interactions which counteract the movement of the grains, and the 
hydrodynamic forces that reduce the weight of the solids inducing a transition 
from dense compacted to a dense suspended flow. Hence, it is important to 
understand the mechanism of underwater granular flows at the granular scale. A 
pending research issue is the parameterisation of interactions between the 
water phase and the sediment phase. Owing to the number of flow variables 
involved and measurement imprecision, estimating such parameters from 
laboratory experiments remains difficult. 

In this study, two-dimensional sub-grain scale numerical simulations are 
performed to understand the local rheology of a dense granular flows in a 
fluid. The Discrete Element Method (DEM) is coupled with the Lattice Boltzmann 
Method (LBM) for fluid-grain interactions, to understand the evolution of 
immersed granular flows. The fluid phase is simulated using 
Multiple-Relaxation-Time LBM method for better numerical stabilities. The 
Eulerian nature of the LBM formulation, together with the common explicit time 
step scheme of both LBM and DEM makes this coupling strategy an efficient 
numerical procedure for systems dominated by both grain--fluid and grain--grain 
interactions. The D2Q9 Model in LBM is used to simulate the fluid phase. In 
order to simulate interconnected pore space in 2D, a reduction in radius of the 
grains is assumed during LBM computations. Granular materials of different 
permeabilities are simulated by varying the reduction in radius of the grains. 
A parametric analysis is performed to assess the influence of the grain sample 
characteristics (initial configuration, permeability, slope of inclined plane) 
on the evolution of flow and run-out distances. The effect of hydrodynamic 
forces and hydroplaning on the run-out evolution is analysed by comparing the 
mechanism of energy dissipation and flow evolution in dry and immersed granular 
flows. Voronoi tesselation was used to study the evolution of local density and 
water entrainment at the flow front. A parameteric analysis is performed to 
assess the influence of the grain sample characteristics (initial 
configuration) and the fluid properties (e.g., viscosity) on the evolution of 
flow and run-out distances. The effect of hydrodynamic forces on the run-out 
evolution is analysed by comparing the mechanism of energy dissipation and flow 
evolution in dry and immersed granular flows.


\end{abstract}
