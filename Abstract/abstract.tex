% ************************** Thesis Abstract *****************************
% Use `abstract' as an option in the document class to print only the titlepage and the abstract.
\begin{abstract}


Geophysical hazards, such as avalanches, debris flows and submarine landslides, 
involve rapid mass movement of granular solids, water and air as a single-phase 
system. The momentum transfer between discrete and continuum phases  
significantly affects the dynamics of the flow. The dynamics of a granular flow 
involve at least three distinct scales:the \textit{microscopic scale}, which is 
characterized by contact between particles, the \textit{meso-scale}, which 
represents micro-structural effects such as particle rearrangement, and the 
\textit{macroscopic scale}. This study aims to understand the ability of 
continuum models in capturing the micro-mechanism of granular flow dynamics. 
The initiation and propagation of granular flows depend mainly on the slope, 
density, and quantity of the material destabilised. 

Collapse of a granular column on a horizontal surface is a simple case of 
granular flow, however, the mechanism of collapse and the flow dynamics is yet 
to be understood. In the present study, multi-scale modelling, i.e. 
discrete-element and continuum analyses, of quasi-two dimensional collapse of 
granular columns are performed. Material Point Method (MPM), a hybrid 
Lagrangian and Eulerian approach is used to describe the continuum behaviour of 
granular column collapse, while the micromechanics is captured using Discrete 
Element Method (DEM) with tangential contact force model. 

The run-out distance obtained exhibit a power law dependency with the aspect 
ratio of the column. Discrete-element approach predicts transition behaviour in 
the run-out distance with increase in the aspect ratio of the granular column. 
The run-out profile predicted by the continuum simulations matches with DEM 
simulations for columns with small aspect ratios (`h/l' $\le$ 2), however MPM 
predicts longer run-out distances for columns with higher aspect ratios (`h/l' 
> 2). Energy evolution study in DEM simulations reveals higher collisional 
dissipation in the initial free-fall regime for tall columns. The lack of 
collisional energy dissipation mechanism in MPM simulations results in longer 
run-out distances. In DEM simulations, Voronoi tesselation is used to evaluate 
the development of shear bands and evolution of local packing density. Studies 
on granular collapse with different initial densities reveal the same critical 
density at the end of the flow. A sliding flow regime is observed above the 
distinct passive zone at the core of the column. Stress profiles obtained from 
both the scales are compared to understand reason for a slow flow run-out 
mobilization in MPM simulations. 

Certain macroscopic models are able to capture simple mechanical behaviours, 
however the complex physical mechanisms that occur at the grain scale, such as 
hydrodynamic instabilities, the formation of clusters, collapse, and transport, 
have largely been ignored. In order to describe the mechanism of saturated 
and/or immersed granular flows, it is important to consider both the dynamics 
of the solid phase and the role of the ambient fluid. In particular, when the 
solid phase reaches a high volume fraction, it is important to consider the 
strong heterogeneity arising from the contact forces between the grains, the 
drag interactions which counteract the movement of the grains, and the 
hydrodynamic forces that reduce the weight of the solids inducing a transition 
from dense compacted to a dense suspended flow. Hence, it is important to 
understand the mechanism of underwater granular flows at the granular scale. A 
pending research issue is the parameterisation of interactions between the 
water phase and the sediment phase. Owing to the number of flow variables 
involved and measurement imprecision, estimating such parameters from 
laboratory experiments remains difficult. 

In this study, two-dimensional sub-grain scale numerical simulations are 
performed to understand the local rheology of a dense granular flows in a 
fluid. The Discrete Element Method (DEM) is coupled with the Lattice Boltzmann 
Method (LBM) for fluid-grain interactions, to understand the evolution of 
immersed granular flows. The fluid phase is simulated using 
Multiple-Relaxation-Time LBM method for better numerical stabilities. The 
Eulerian nature of the LBM formulation, together with the common explicit time 
step scheme of both LBM and DEM makes this coupling strategy an efficient 
numerical procedure for systems dominated by both grain--fluid and grain--grain 
interactions. The D2Q9 Model in LBM is used to simulate the fluid phase. In 
order to simulate interconnected pore space in 2D, a reduction in radius of the 
grains is assumed during LBM computations. Granular materials of different 
permeabilities are simulated by varying the reduction in radius of the grains. 
A parametric analysis is performed to assess the influence of the grain sample 
characteristics (initial configuration, permeability, slope of inclined plane) 
on the evolution of flow and run-out distances. The effect of hydrodynamic 
forces and hydroplaning on the run-out evolution is analysed by comparing the 
mechanism of energy dissipation and flow evolution in dry and immersed granular 
flows. Voronoi tesselation was used to study the evolution of local density and 
water entrainment at the flow front. A parameteric analysis is performed to 
assess the influence of the grain sample characteristics (initial 
configuration) and the fluid properties (e.g., viscosity) on the evolution of 
flow and run-out distances. The effect of hydrodynamic forces on the run-out 
evolution is analysed by comparing the mechanism of energy dissipation and flow 
evolution in dry and immersed granular flows.


\end{abstract}
