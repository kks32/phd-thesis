% ************************** Thesis Abstract *****************************
% Use `abstract' as an option in the document class to print only the titlepage and the abstract.
\begin{abstract}


Geophysical hazards usually involve flow of dense granular solids and water as 
a single-phase system. Understanding the mechanics of granular flow is of 
particular importance in predicting the run-out distances of debris flows. The 
dynamics of a homogeneous granular flow involve at least three distinct scales: 
the \textit{microscopic scale}, which is characterised by contact between 
grains, \textit{the meso-scale} that represents micro-structural effects such 
as grain rearrangement, and the \textit{macroscopic scale}, where geometric
correlations can be observed. 

Conventionally, granular materials such as soils are modelled as a continuum. 
On a macroscopic scale, granular materials exhibit many collective phenomena 
and the use of continuum mechanics to describe the macroscopic behaviour can be 
justified. However, on a grain scale, the granular materials exhibit complex 
solid-like and/or fluid-like behaviour depending on how the grains interact 
with each other. Recent works on granular materials suggest that a continuum 
law may be incapable of revealing inhomogeneities at the grain scale level, 
such as orientation of force chains, which are purely due to micro-structural 
effects. Discrete element method (DEM) is capable of simulating the granular 
material as a discontinuous system allowing one to probe into local variables 
such as position, velocities, contact forces, etc. In the present study, a 
multi-scale approach is adopted to better understand the rheology of granular 
flows and the limitations of continuum models.

The collapse of a granular column on a horizontal surface is a simple case of 
granular flow, however a proper model that describes the flow dynamics is still
lacking. Granular flow is modelled as a frictional dissipation process in 
continuum mechanics but studies showing the lack of influence of inter-particle 
friction on the energy dissipation and spreading dynamics is surprising. In the 
present study, the generalised interpolation material point method (GIMPM), a 
hybrid Eulerian -- Lagrangian approach, is implemented with 
Mohr-Coloumb failure criterion to describe the continuum behaviour of granular 
flows. While, the micro-mechanics of granular flows is modelled using DEM.

The run-out distance of a granular column collapse exhibits a power law 
dependency with the aspect ratio of the column. The difference between the 
continuum and discrete approaches in modelling the collapse and spreading 
dynamics is studied by inspecting the energy dissipation mechanisms. The lack 
of collisional dissipation in MPM is found to be the primary reason for the 
discrepancy in modelling collapse of tall granular columns. The classical 
Mohr-Coloumb model has the ability to capture the rheology of granular flows  
in dense-granular and critical state flow regimes, such as run-out evolution of 
slopes subjected to impact loading, where the inertial number \textit{I} < 0.1. 

The initiation and propagation of submarine granular flows depend mainly on the 
slope, density, and quantity of the material destabilised. The
complex physical mechanisms that occur at the grain scale, such as the 
hydrodynamic instabilities and formation of clusters, have largely been 
ignored. A GPU paralellised two-dimensional Lattice Boltzmann LBM -- DEM 
coupled technique is developed to understand the local rheology of dense 
granular flows in fluid. Granular materials of different permeabilities are 
simulated by varying the hydrodynamic radius of the grains. Parametric 
analyses are performed to assess the influence of the initial configuration, 
permeability, and the slope of the inclined plane on the 
evolution of flow and run-out distances. The effect of hydrodynamic forces and 
hydroplaning on the run-out evolution is analysed by comparing the mechanism of 
energy dissipation and the flow evolution in dry and immersed granular flows. 
Voronoi tesselation is used to capture the meso-scale behaviour such as the 
evolution of local density and water entrainment at the flow front. 


Certain macroscopic models are able to capture simple mechanical behaviours, 
however the complex physical mechanisms that occur at the grain scale, such as 
hydrodynamic instabilities, the formation of clusters, collapse, and transport, 
have largely been ignored. In order to describe the mechanism of saturated 
and/or immersed granular flows, it is important to consider both the dynamics 
of the solid phase and the role of the ambient fluid. In particular, when the 
solid phase reaches a high volume fraction, it is important to consider the 
strong heterogeneity arising from the contact forces between the grains, the 
drag interactions which counteract the movement of the grains, and the 
hydrodynamic forces that reduce the weight of the solids inducing a transition 
from dense compacted to a dense suspended flow. Hence, it is important to 
understand the mechanism of underwater granular flows at the granular scale. A 
pending research issue is the parameterisation of interactions between the 
water phase and the sediment phase. Owing to the number of flow variables 
involved and measurement imprecision, estimating such parameters from 
laboratory experiments remains difficult. 

In this study, two-dimensional sub-grain scale numerical simulations are 
performed to understand the local rheology of a dense granular flows in a 
fluid. The Discrete Element Method (DEM) is coupled with the Lattice Boltzmann 
Method (LBM) for fluid-grain interactions, to understand the evolution of 
immersed granular flows. The fluid phase is simulated using 
Multiple-Relaxation-Time LBM method for better numerical stabilities. The 
Eulerian nature of the LBM formulation, together with the common explicit time 
step scheme of both LBM and DEM makes this coupling strategy an efficient 
numerical procedure for systems dominated by both grain--fluid and grain--grain 
interactions. The D2Q9 Model in LBM is used to simulate the fluid phase. In 
order to simulate interconnected pore space in 2D, a reduction in radius of the 
grains is assumed during LBM computations. Granular materials of different 
permeabilities are simulated by varying the reduction in radius of the grains. 
A parametric analysis is performed to assess the influence of the grain sample 
characteristics (initial configuration, permeability, slope of inclined plane) 
on the evolution of flow and run-out distances. The effect of hydrodynamic 
forces and hydroplaning on the run-out evolution is analysed by comparing the 
mechanism of energy dissipation and flow evolution in dry and immersed granular 
flows. Voronoi tesselation was used to study the evolution of local density and 
water entrainment at the flow front. A parameteric analysis is performed to 
assess the influence of the grain sample characteristics (initial 
configuration) and the fluid properties (e.g., viscosity) on the evolution of 
flow and run-out distances. The effect of hydrodynamic forces on the run-out 
evolution is analysed by comparing the mechanism of energy dissipation and flow 
evolution in dry and immersed granular flows.

\end{abstract}
