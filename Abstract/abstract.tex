% ************************** Thesis Abstract *****************************
% Use `abstract' as an option in the document class to print only the titlepage and the abstract.
\begin{abstract}


Geophysical hazards usually involve flow of dense granular solids and water as 
a single-phase system. The dynamics of a homogeneous granular flow involve 
three distinct scales: the \textit{microscopic scale}, the \textit{meso-scale}, 
and the \textit{macroscopic scale}. Granular materials exhibit complex 
solid-like and/or fluid-like behaviour depending on how the grains interact 
with each other. Although soil flows are conventionally modelled as a 
continuum, recent studies have shown the limitations of continuum models to 
capture the in-homogeneities at the grain-scale level. In the present study, a 
multi-scale approach is adopted to understand the rheology of granular flows 
and the limitation of continuum models.

The Generalised Interpolation Material Point Method (GI-MPM), a  hybrid 
Eulerian -- Lagrangian approach is implemented in the present study to describe 
the continuum behaviour of granular flows. The Discrete Element Method (DEM) is 
used to model the micro-mechanics of granular flows. A two-dimensional 
collapse of a granular column on a horizontal surface is studied. The run-out 
distance of a granular column collapse exhibits a power law dependency with the 
aspect ratio of the column. For columns with small aspect ratios (`h/l' $\le$ 
2), both the approaches predict similar run-out behaviour. However, MPM 
predicts longer run-out distances for columns with larger aspect ratios (`h/l' 
> 2). The energy dissipation mechanism during the collapse reveals higher 
collisional dissipation in the initial free-fall regime for tall columns. The 
lack of collisional dissipation in MPM is found to be the reason for longer 
run-out distances for tall columns. The classical Mohr-Coloumb model has the 
ability to capture the rheology of granular flows  in dense-granular 
and critical state flow regimes, such as run-out evolution of slopes 
subjected to impact loading, where the Inertial number \textit{I} < 0.1. 

The initiation and propagation of submarine granular flows depend mainly on the 
slope, density, and quantity of the material destabilised. The
complex physical mechanisms that occur at the grain scale, such as the 
hydrodynamic instabilities and formation of clusters, have largely been 
ignored. A GPU paralellised two-dimensional Lattice Boltzmann LBM -- DEM 
coupled technique is developed to understand the local rheology of a dense 
granular flows in a fluid. Granular materials of different permeabilities are 
simulated by varying the hydrodynamic radius of the grains. A parametric 
analysis is performed to assess the influence of the initial configuration, 
permeability, and the slope of the inclined plane on the 
evolution of flow and run-out distances. The effect of hydrodynamic forces and 
hydroplaning on the run-out evolution is analysed by comparing the mechanism of 
energy dissipation and the flow evolution in dry and immersed granular flows. 
Voronoi tesselation is used to capture the meso-scale behaviour such as the 
evolution of local density and water entrainment at the flow front. 


\end{abstract}
